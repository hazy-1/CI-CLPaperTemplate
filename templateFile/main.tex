\documentclass[a4paper,11truept, titlepage]{jlreq}
\usepackage{styles}

\titleJP{CI学科、CLのための論文}
\titleEN{Dissertation for the Department of Creative Innovation or Creative Leadership.}
\info{C1121502}{角田 創}
% 院生のみ適用する項目
% \mainChecker{hoge}
% \subChecker{fuga}

\begin{document}
\maketitle
\mojiparline{35} % 一行あたり文字数の指定
\linesparpage{30} % 1ページあたり行数の指定

\section{はじめに}
これは武蔵野美術大学クリエイティブイノベーション学科、クリエイティブリーダーシップの共通論文フォーマットである。
図\ref{abstract}は、このように挿入される。
\begin{figure}
    \centering
	\includegraphics[clip, width=\columnwidth]{Images/test.png}
    \caption{ここにキャプションがきます。}
    \label{abstract}
\end{figure}

もちろん、参考文献についてもこのように扱われる。\citep{bib1}

\subsection{サブセクション}
サブセクションについても当然扱われる。

\small
\bibliographystyle{jecon}
\bibliography{reference}
\end{document}